
% Work experience

\part{Work experience}

\entry{Oct 2019}{Present}{Semantic Web technologist}{RENCI, UNC Chapel Hill, North Carolina}

At the Renaissance Computing Institute (RENCI) at the University of North Carolina (UNC), I work on several projects to facilitate the re-use of biomedical data by other scientists, with a particular focus on using Semantic Web technologies to make scientific data findable, accessible, interoperable, and reusable (FAIR). A key challenge with using the Semantic Web in scientific applications is that many biomedical resource use different identifiers for similar or identically defined concepts. This is a problem I have a lot of experience with, as I first started working on this challenge in during my PhD on biological taxonomy, as the same biological taxa may be assigned different identifiers in different taxonomic databases, or multiple biological taxa may be referred to by the same biological taxa.

To tackle this challenge, I help build databases that document how identifiers in one resource relate to identifiers in other resources. In addition, I work on tools that make it easier for anybody to use these mappings and their associated resources. The tools I work on use standards such as RDF and JSON-LD to store and transmit semantically rich data, use ontologies written in OWL to reason over data, and use descriptions in SHACL and ShEx to validate data and document data models. Where existing tools are unavailable or inadequate, I write software tools to fill this need. My current projects over almost every time of biological concept, from biological taxa to genes and proteins to biochemical pathways to diseases and phenotypes to small molecules and drugs. I work primarily in Python and Scala to build these tools.

\begin{itemize}

\item \textit{NCATS Biomedical Data Translator} (\since{2022}): The Translator project is an ambitious project to build a tool capable of answering complex biomedical questions by querying multiple biomedical search engines (``Autonomous Relay Agents'') developed for this purpose, each of which can itself query a large number of knowledge sources. These results are synthesized into a single list of potential solutions with citations to appropriate publications and presented to the user. However, different Translator components -- particularly the knowledge sources -- could use different identifiers to refer to the same biomedical concepts.

In order to create a single set of normalized concept identifiers, RENCI previously developed a Node Normalization tool, a massive database of biomedical identifier cliques that are believed to refer to the same concept, with a single preferred identifier for each clique. We also developed a Name Resolution tool, which allows this database of cliques to be searched for by names and synonyms, which is useful for both recognizing biomedical concepts mentioned in text and for enabling autocomplete functionality on Translator websites.

Since 2022, I have been in charge of both of these components, as well as a developer on one of the knowledge sources (Causal Activity Models Knowledge Provider, or CAM-KP) that provides information on biomedical pathways to the Translator search engines. I work on triaging issues reported by scientists working on Translator and fixing bugs to ensure that these components are as accurate as possible, developing new features such as improved search queries and adding descriptions from ontologies, and improving how these components are built and deployed on both our in-house Kubernetes cluster as well as production cluster hosted on the NCATS servers.

\begin{products}

\product[2]{Babel, a Snakemake-based Python data pipeline for generating the biomedical concept clique information: \github{TranslatorSRI}{Babel}.}

\product[2]{Node Normalization, a Python web app for normalizing biomedical concept identifiers and Translator API (TRAPI) messages: \github{TranslatorSRI}{NodeNormalization}.}

\product[2]{Name Resolution, a Python web app for searching for biomedical concept identifiers using names and synonyms: \github{TranslatorSRI}{NameResolution}.}

\product[2]{Babel Validation (started by me), a set of Scala and Vue/JavaScript tools for validating Babel cliques and Node Normalization/Name Resolution endpoints: \github{TranslatorSRI}{babel-validation}.}

\product[2]{CAM-Pipeline, a data pipeline using Make, Scala and Souffle to convert biomedical pathway information from a variety of data sources into the TRAPI format for use by Translator applications: \github{ExposuresProvider}{cam-pipeline}.}

\product[2]{CAM-KP-API (\fromto{2022}{2023}), a Scala web app for making CAM-KP biomedical pathway information accessible via TRAPI. Now replaced with the ORION pipeline and Plater tool, but source code still available: \github{ExposuresProvider}{cam-kp-api}.}

\end{products}

\item \textit{NIH HEAL Data Stewards} (\since{2021}): The National Institutes of Health Helping to End Addiction Long-term Initiative (``NIH HEAL Initiative'') is an aggressive, trans-agency effort to speed scientific solutions to stem the national opioid and pain public health crises. The HEAL Data Stewards are the component of HEAL responsible for ensuring that all the scientific data collected from HEAL-funded studies are as Findable, Accessible, Interoperable and Reusable (FAIR) as possible. Apart from assisting with data standardization efforts, most of my work on HEAL focused on building HEAL Semantic Search, based upon a semantic search engine previously developed at RENCI called Dug. I work on identifying and designing features needed to meet HEAL's scientific needs, developing data ingest and annotation tools for a variety of scientific data, and work on incorporating new types of resources into HEAL Semantic Search, such as information from the HEAL Common Data Elements (CDE) Repository.

\begin{products}

\product[2]{HEAL Semantic Search, a search engine for HEAL data: \url{https://heal.renci.org/}, \github{helxplatform}{dug}.}

\end{products}

\item \textit{FHIRCat} (\since{2021}): The FHIR Catalog (FHIRCat) project aims to simplify the use of the RDF representation of the Fast Health Interoperability Resources (FHIR) specification so that FHIR data can be aggregated and analyzed using triple stores. This involves discussions with FHIR developers to come up with the best ways of representing FHIR in RDF, testing these approaches on actual patient data, and building software tools for filling in any gaps in these approaches.

\item \textit{Center for Cancer Data Harmonization} (\fromto{2019}{2022}): I was a member of the Tools and Data Quality group at CCDH, where I am worked with other programmers and scientists to build model description, data validation and transformation tools for a variety of biomedical data related to cancer. Among the types of data that CCDH hoped to harmonize include genomic, proteomic, image and patient data, requiring flexible tools that can understand a variety of information types. I worked primarily on two tools:

\begin{products}

\product[2]{csv2caDSR, a tool for harmonizing CSV files against common data elements in the caDSR: \url{https://github.com/cancerDHC/csv2caDSR}}

\product[2]{umls-rrf-scala, a tool for generating mappings between terms in the NCI Metathesaurus in the UMLS RRF format: \url{https://github.com/cancerDHC/umls-rrf-scala}}

\end{products}

\begin{comment}

\item \textit{Data modeling and validation with ClinGen} (2019): The ClinGen project is working on developing several data models for recording and sharing information relating to rare genetic variants that may be involved in causing diseases. The current spreadsheet-based data model description requires manual translation into the JSON Schema language to validate data against this specification. I am working with a group of developers to convert the data model into the SHACL shape description language, which could be used both to validate new data and to produce human-readable documentation.

\begin{products}

\product[2]{SHACLI, a command-line tool for validating RDF data against SHACL shapes: \url{https://shacli.org/}, with source code available at \url{https://github.com/gaurav/shacli}}

\end{products}

\end{comment}

\item \textit{Reasoning Over Biomedical Objects linked in Knowledge Oriented Pathways} (Robokop): I am improving an existing tool for extracting useful information from PubMed's database of publication abstracts\footnote{Omnicorp, source code available at \url{https://github.com/NCATS-Gamma/omnicorp}}. My contribution has been in adding publication metadata, such as authorship, publication information and DOIs, in RDF for use by downstream graph databases, as well as generally improving testing and deployment options. It is also an opportunity for me to improve my skills with running long-running tasks on computer clusters.


\begin{products}

\product[2]{Korn D, Bobrowski T, Li M, Kebede Y, Wang P, Owen P, \textbf{Vaidya G}, Muratov E, Chirkova R, Bizon C, Alexander Tropsha A (2020) COVID-KOP: integrating emerging COVID-19 data with the ROBOKOP database. \textit{Bioinformatics}:btaa718 \doi{10.1093/bioinformatics/btaa718}}

\end{products}

\end{itemize}

\entry{2018}{2019}{Postdoctoral Associate}{Florida Museum of Natural History, Gainesville}

I worked full-time as the lead software developer on the Phyloreferencing project (\href{http://phyloref.org}{phyloref.org}). Our goal is to build an ontology of definitions for groups of related biological organisms, as well as the software and ontological infrastructure needed to create, edit, organize and test these definitions. We are building a demonstration website that will allow users to resolve these definitions on any evolutionary hypothesis. I built these tools using JavaScript in Node.js, Vue CLI, Java and Python.

\begin{products}

\product[2]{Source code, issues and project management available at \url{https://github.com/phyloref}.}

\product[2]{\textit{Phyx.js} package published to NPM as \url{https://npmjs.com/package/@phyloref/phyx}.}

\product[2]{\textbf{Vaidya, G*}, Cellinese N, Lapp H (2022) A new phylogenetic data standard for computable clade definitions: the Phyloreference Exchange Format (Phyx). PeerJ 10: e12618. \doi{10.7717/peerj.12618}.}

\product[2]{\textbf{Vaidya, G*}, Cellinese N, Lapp H (2021) JPhyloRef: a tool for testing and resolving phyloreferences. Journal of Open Source Software, 6(64), 3374. \doi{10.21105/joss.03374}.}

\product[2]{\textbf{Vaidya, G*}, Lapp H, Cellinese N (October 22, 2020) Enabling Machines to Integrate Biodiversity Data with Evolutionary Knowledge. Presentation at the \textit{TDWG 2020} conference, online. \\
\doi{10.3897/biss.4.59088}.}

\product[3]{\textbf{Vaidya, G*}, Lapp H, Cellinese N (March 12, 2019) Building an ontology of logic definitions for groups of biological organisms to enable data integration. Poster at \textit{US2TS 2019}, Durham NC, USA. \doi{10.6084/m9.figshare.7904999.v1}.}

\product[2]{\textbf{Vaidya, G*}, Lapp H, Cellinese N* (August 29, 2018) All the Clades in the World: Building a Semantically-Rich and Testable Ontology of Phylogenetic Clade Definitions. Presentation at the \textit{SPNHC+TDWG 2018} conference, Dunedin, New Zealand. \doi{10.3897/biss.2.25776}.}

\end{products}

I also worked part-time on the Phyloreferencing project while completing my PhD (2016 to 2017) and for a short time while at RENCI under a subcontract (2019 to 2020).

\begin{products}

\product[2]{\textbf{Vaidya G*}, Lapp H, Cellinese N (December 6, 2016) Creating computable definitions for clades using the Web Ontology Language (OWL). Presentation at \textit{TDWG 2016 Annual Conference}, Santa Clara de San Carlos, Costa Rica. (\href{https://speakerdeck.com/gaurav/creating-computable-definitions-for-clades-using-the-web-ontology-language-owl}{slides}, \href{https://vimeo.com/198739085}{video})}

\product[2]{\textbf{Vaidya G*}, Lapp H, Cellinese N (June 18, 2016) The Semantic Clade. Presentation at \textit{Evolution 2016}, Austin TX. (\href{https://speakerdeck.com/gaurav/the-semantic-clade}{slides}, \href{https://www.youtube.com/watch?v=_aNaAQYTNVc}{video})}

\end{products}

\entry{2015}{2016}{Graduate teaching assistant}{University of Colorado Boulder}

While a graduate student, I taught three Evolutionary Biology and General Biology labs, in which I led classes of up to eighteen undergraduate students through hands-on exercises to build their understanding of biology and evolution. CU Boulder asks students to evaluate their instructors on a scale from $1$ (worst) to $6$ (best); my reviews increased from $4.8-5.5$ in my first semester to $5.3-5.6$ in my last semester.

\entry{2011}{2015}{Graduate research/teaching assistant}{University of Colorado Boulder}

During the first four years of my PhD, I worked on the Map of Life and VertNet projects, where I developed a web application for synthesizing and managing vernacular names extracted from multiple sources\footnote{Source code available at \url{https://github.com/MapofLife/vernacular-names}}, a web API for efficiently searching a large database of taxonomic names\footnote{Source code available at \url{https://github.com/gaurav/taxrefine}}, and a Python tool to identify errors in species names (now deprecated).

\begin{products}

\product[2]{Vaidya G (July 22, 2013) Validating scientific names with the GBIF Checklist Bank. Blog post at \url{https://blog.vertnet.org/post/56169017224/taxonomic-validation-vaidya}.}

\product[3]{Wieczorek J*, Steele A, Jetz W, Guralnick R, Hill A, \textbf{Vaidya G*}, Rosauer D (October 20, 2011) Map of Life. Presentation at \textit{TDWG 2011}, New Orleans LA. (\href{https://speakerdeck.com/gaurav/map-of-life-computer-demo-at-tdwg-2011}{slides})}

\end{products}

\entry{May}{August 2014}{Student developer}{\mbox{DBpedia}, funded by Google Summer of Code}

I extended \mbox{DBpedia's} fact extraction software to support extracting facts in RDF from the Wikimedia Commons, an online repository that then contained around 25~million media files across a number of formats and licenses.

\begin{products}

\product[3]{\textbf{Vaidya G*} (October 9, 2015) Metadata from the Commons. Presentation at \textit{WikiConference USA 2015}, Washington DC.}

\product[1]{\textbf{Vaidya G}, Kontokostas D, Knuth M, Lehmann J, Hellmann S (2015) DBpedia Commons: Structured multimedia metadata from the Wikimedia Commons. \textit{The Semantic Web --- ISWC 2015}, volume \textbf{9367} of Lecture Notes in Computer Science, pages 281-289. Springer International Publishing. \doi{10.1007/978-3-319-25010-6\_17}.}

\end{products}

\entry{January}{May 2013}{Graduate fellowship}{NESCent, Durham, North Carolina}

\nopagebreak
This fellowship allowed me to work exclusively on my PhD for one semester with a mentor at the National Evolutionary Synthesis Center (NESCent) in Durham, North Carolina\footnote{Proposal available at \url{https://www.nescent.org/science/awards\_summary.php-id=374.html}}. My mentor at NESCent, Hilmar Lapp, would later hire me to work on the Phyloreferencing project.

\entry{June}{August 2012}{Graduate student assistant}{University of Colorado Boulder}

I worked on the Art of Life project\footnote{Details at \url{https://about.biodiversitylibrary.org/projects/past-projects/art-of-life/}} funded by the Missouri Botanical Garden, where I helped develop and test a data schema to store structured information on illustrations, photographs and maps extracted from biodiversity journals.

\begin{products}

\product[2]{\textbf{Vaidya G*} (March 13, 2015) Describing natural history illustrations in the Art of Life project via the Wikipedia Commons platform. Presentation at \textit{Visual Resources Association's 33rd Annual Conference}, Denver CO. (\href{https://speakerdeck.com/gaurav/vra-2015-describing-natural-history-illustrations-in-the-art-of-life-project-via-the-wikipedia-commons-platform}{slides})}

\end{products}

\entry{2007}{2011}{Software architect}{Paper Terminal Pte Ltd, Singapore}

I was the primary software developer, responsible for developing new web applications from prototypes to final deployment, including {\it OCR~Terminal}, my company's flagship product. I also managed our computer systems on both on-site and Amazon EC2 cloud plaform across multiple operating systems.

\entry{2006}{2007}{Lab officer}{Evolutionary Biology Laboratory, National University of Singapore}

I helped manage computer-related infrastructure, from sending computers for servicing to installing scientific software on both local hardware and remote computing clusters. I also finished work on several scientific tools, which I have documented under my educational history below.