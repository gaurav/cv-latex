% Educational qualifications

\part{Educational history}

\entry{2011}{2017}{Doctor of Philosophy}{University of Colorado Boulder, USA}

My dissertation in Ecology and Evolutionary Biology was supervised by Prof. Robert Guralnick. I quantified the rates at which scientific names and their meanings changed within two North American taxonomic checklist series, each made up of dozens of checklists containing hundreds of taxonomic changes. I built a software tool in Java that I used to identify and annotate these changes, facilitating their use as biodiversity data\footnote{SciNames, source code available at \url{https://github.com/gaurav/scinames}}.

\begin{products}

\product[1]{\textbf{Vaidya G}, Lepage D, Guralnick R (2018) The tempo and mode of the taxonomic correction process: how taxonomists have corrected and recorrected North American bird species over the last 127 years. \textit{PLOS ONE} \textbf{13}(4):~e0195736 \doi{10.1371/journal.pone.0195736}.}

\product[1]{\textbf{Vaidya G} (2017) Taxonomic Checklists as Biodiversity Data: How Series of Checklists can Provide Information on Synonymy, Circumscription Change and Taxonomic Discovery. \textit{Ph.D. Dissertation}. \doi{10.17605/OSF.IO/N2KH5}.}

\product[3]{\textbf{Vaidya G*}, Guralnick R (October 1, 2014) Names, meanings, and the meaninglessness of names. \textit{The Meaning of Names: Naming Diversity in the 21st Century}, University of Colorado Museum of Natural History, Boulder CO.}

\product[1]{\textbf{Vaidya G*}, Lepage D, Lapp H, Guralnick RP (September 26, 2014) Measuring The Outputs Of Taxonomy: How Many Species Of North American Birds Have Been Recircumscribed In The Last 128 Years? Presentation at \textit{AOU/COS/SCO Annual Meeting 2014}, Estes Park, Colorado.}

\product[1]{Lepage D, \textbf{Vaidya G}, Guralnick R (2014) Avibase --- a database system for managing and organizing taxonomic concepts. \textit{ZooKeys}~\textbf{420}:~117-135. \doi{10.3897/zookeys.420.7089}}

\product[2]{\textbf{Vaidya G*}, Lepage D, Lapp H, Guralnick R (August 14, 2013) Quantifying taxonomic redescription: patterns of lumping and splitting in the last 127 years of the \textit{Check-List of North American Birds}. Presentation at \textit{AOU/COS Annual Meeting 2013}, Chicago IL. (\href{https://speakerdeck.com/gaurav/quantifying-taxonomic-redescription-patterns-of-lumping-and-splitting-in-the-last-127-years-of-the-check-list-of-north-american-birds}{slides})}

\product[2]{Stoltzfus A, et al. (2013) Phylotastic! Making tree-of-life knowledge accessible, reusable and convenient. \textit{BMC Bioinformatics}~\textbf{14}:~158. \doi{10.1186/1471-2105-14-158}.}

\product[2]{Thomer A, \textbf{Vaidya G*}, Guralnick R, Bloom D, Russell L (November 8, 2012) Extracting data from historical documents. Presentation at \textit{Museum Computer Network 2012}, Seattle WA. (\href{http://www.slideshare.net/mrvaidya/extracting-data-from-historical-documents-crowdsourcing-annotations-on-wikisource}{slides}, \href{http://www.youtube.com/watch?v=_FHJgI1Aj0g}{video})}

\product[2]{Thomer A, \textbf{Vaidya G}, Guralnick R, Bloom D, Russell L (2012) What Henderson Saw: Extracting observations from century-old field notebooks. \textit{ZooKeys} \textbf{209}:~235-253. \\ \doi{10.3897/zookeys.209.3247}.}

\end{products}

\entry{2002}{2006}{Bachelor of Science with Merit}{National University of Singapore}

I majored in Life Sciences with minors in Computational Science and Economics. I began working with the Evolutionary Biology lab in my first year as an undergraduate, and was hired as a lab officer in my first year after graduating. Several methods of identifying specimens using DNA sequencing of a small number of genes (``DNA barcoding'') had been proposed, and I developed software tools in Java that allowed my lab to measure their effectiveness on real-world data downloaded from the NCBI GenBank database. I also wrote a software tool to make it easier for scientists to assemble multi-gene, multi-species datasets of genetic database.

\begin{products}

\product[3]{Kwong S, Srivathsan A, \textbf{Vaidya G}, Meier R (2011) Is the COI barcoding gene involved in speciation through intergenomic conflict? \textit{Molecular Phylogenetics and Evolution} \textbf{62}(3):~1009-1012. \doi{10.1016/j.ympev.2011.11.034}.}

\product[1]{\textbf{Vaidya, G}, Lohman D, Meier R (2011) SequenceMatrix: concatenation software for the fast assembly of multi-gene datasets with character set and codon information. \textit{Cladistics}, \textbf{27}:~171-180. \doi{10.1111/j.1096-0031.2010.00329.x}.}

\product[2]{\textbf{Vaidya G*}, Meier R, Lohman D (June 26, 2009) SequenceMatrix: Gene concatenation made easy. Presentation at \textit{Hennig~XXVIII}, Singapore.}

\product[1]{Dikow T, Meier R, \textbf{Vaidya GG}, Londt JGH (2009) Biodiversity research based on taxonomic revisions --- a tale of unrealized opportunities. \textit{Diptera Diversity: Status, Challenges and Tools}, chapter 12. Koninklijke Brill NV. \url{http://www.tdvia.de/pdf/dikow_etal_2009.pdf}.}

\product[1]{Meier R, Kwong S, \textbf{Vaidya G}, Ng PKL (2006) DNA Barcoding and Taxonomy in Diptera: A Tale of High Intraspecific Variability and Low Identification Success. \textit{Systematic Biology} \textbf{55}(5): 715-728. \doi{10.1080/10635150600969864}.}

\end{products}
