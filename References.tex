\documentclass[12pt,letter]{article}

\usepackage{lastpage}
\usepackage{changepage}   % for the adjustwidth environment
\usepackage[hang,flushmargin]{footmisc}
\usepackage{todonotes}
\usepackage{tgtermes}
\usepackage{helvet}
\usepackage{marvosym}
\usepackage{lipsum}
\usepackage{enumitem}
\usepackage{anysize}
\usepackage{wrapfig}
\usepackage{pifont}
\usepackage{comment}
\usepackage[small,compact]{titlesec}
\usepackage[colorlinks=true, urlcolor=blue, linkcolor=black]{hyperref}

%\marginsize{28mm}{28mm}{8mm}{8mm}
%\marginsize{22mm}{22mm}{4mm}{4mm}
\marginsize{24mm}{24mm}{8mm}{4mm}
\setlength{\parindent}{0mm}
\setlength{\parskip}{1em}
\titlespacing{\subsection}{0pt}{*0}{*0}

\newenvironment{publications}%
    {\small\vspace{-0.5em}\begin{itemize}[noitemsep,nolistsep]}%
    {\end{itemize}\normalsize}

\newcommand{\smallbullet}{$\vcenter{\hbox{\small$\bullet$}}$}
\newcommand{\tinybullet}{$\vcenter{\hbox{\tiny$\bullet$}}$}

\setlist[itemize]{leftmargin=2em}
\renewcommand\labelitemi{\smallbullet}
\renewcommand\labelitemii{\tinybullet}

\renewcommand{\part}[1]{
    {\fontfamily{phv}\selectfont
        \subsection*{\centering #1}
    }
}
\newcommand{\entry}[4]{{\fontfamily{phv}\selectfont\small #1 to #2~\smallbullet~\textbf{#3}~\smallbullet~#4}}

\newcommand{\phonenum}[3]{
    \Telefon~{\tt 
        \href{tel:+#1#2#3}{+1\hspace{0.3em}(#1)\hspace{0.3em}#2\hspace{0.3em}#3}
    }
}
\newcommand{\email}[1]{
    \Letter~{\tt \href{mailto:#1}{#1}}
}
\newcommand{\doi}[1]{\href{https://doi.org/#1}{doi:#1}}

\begin{document}

\pagenumbering{gobble}

% \input{header}

{\fontfamily{phv}\selectfont
    \section*{\large \centering References for Gaurav Vaidya, Ph.D.}
}

\subsection*{Robert Guralnick~\smallbullet~Ph.D. advisor from 2011 to 2017}

Rob was my Ph.D. advisor from the time I started graduate school in 2011 until I defended in fall 2017. As my Ph.D. advisor, Rob has seen me at my best and my worst over the six years that we worked together. He encouraged me to make scientific tool development a core part of my dissertation, and to spend an entire chapter in my dissertation describing the scientific tool I developed to build and analyse the XML datasets that formed the basis of my dissertation. I worked with Rob on several other projects as his graduate student, including a project to transcribe handwritten observation data using Wikisource and then to extract it for publication, developing a data schema for publishing metadata for Biodiversity Heritage Library images on the Wikimedia Commons and Flickr, and a vernacular name database for the Map of Life.

\begin{adjustwidth}{1em}{}
Associate Curator of Biodiversity Informatics \\
Department of Natural History and the Florida Museum of Natural History \\
Dickinson Hall \#358, Gainesville, Florida 32611, USA \\
\url{http://sites.google.com/site/robgur/} \\
\email{rguralnick@flmnh.ufl.edu}
\phonenum{352}{273}{1980}
\end{adjustwidth}

\vspace{0.5em}

\subsection*{Hilmar Lapp~\smallbullet~Supervisor on my current postdoc project}

Hilmar is co-Principal Investigator (co-PI) on the Phyloreferencing project I currently work on, but we first met in 2013 when he mentored me for a semester during my graduate fellowship at the National Evolutionary Synthesis Center (NESCent). Hilmar leads the ontology development part of this project, and I have learned a lot about RDF, linked open data and especially ontology development in the Web Ontology Language (OWL) from him. He also reviews all the software code I write for this project, providing me with valuable feedback on my programming and software engineering practices.

\begin{adjustwidth}{1em}{}
Director of Informatics, Computational Solutions Shared Resource \\
Duke Center for Genomic and Computational Biology \\
107 North, 101 Science Drive, Durham, NC 27708, USA \\
\url{https://genome.duke.edu/directory/gcb-staff/hilmar-lapp} \\
\email{hlapp@duke.edu}
\phonenum{919}{613}{4661}
\end{adjustwidth}

\vspace{0.5em}

\subsection*{Nico Cellinese~\smallbullet~Supervisor on my current postdoc project}

Nico is the other co-PI on the Phyloreferencing project, and is my direct supervisor at the Florida Museum of Natural History. With their previous experience in managing scientific software development projects, Nico and Hilmar gave me a lot of latitude to plan and design the tools we build, estimate my development schedule and to demonstrate my tools to scientists, both in small groups or at scientific conferences. Through working with Nico, I have become better at designing and building software tools that are easy to use, visually compelling and that focus on the precise needs of the scientists we hope to help.

\begin{adjustwidth}{1em}{}
Associate Curator, Herbarium \& Informatics, Florida Museum of Natural History, and \\
Joint Associate Professor, Department of Biology, University of Florida \\
Dickinson Hall \#354, 1659 Museum Road, Gainesville, Florida 32611, USA \\
\url{https://www.floridamuseum.ufl.edu/museum-voices/nico-cellinese/} \\
\email{ncellinese@flmnh.ufl.edu}
\phonenum{352}{273}{1979}
\end{adjustwidth}

\begin{comment}
\begin{center}

\vspace{1em}

\small

\textit{Prepared on \today.}

\end{center}
\end{comment}

\end{document}
